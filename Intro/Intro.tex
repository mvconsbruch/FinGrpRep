\documentclass[../main.tex]{subfiles}
\begin{document}

Let $K$ be a field of characteristic $0$, let $G$ be a finite group of order $d$
and let $V$ be a $K$-vector space on which $G$ acts. In previous talks, we have 
seen a proof of the following theorem.

\begin{thm}[Noether's degree bound] \label{thm:classical}
    In the situation above, the action of $G$ of $K[V] = \Sym(V^*)$ has $K[V]^G$ generated
    in degree $\leq d$, where $d$ is the order of $G$.
\end{thm}

In Daniel's talk, we have even seen that this statement can be generalized a bit,
it holds under the milder assumption that $d < \chr K$. 

Today, we want to investigate further generalizations of this result. 
For one, we want to explain an approach by Derksen \cite{DERKSENProofIdea} that
yields Noether's degree bound under the even milder assumption that $d \in
K^\times$. Afterwards, we explain how this approach makes it possible to
proof an analogue of Noether's degree bound for the exterior algebra.

Perhaps we should recall the definition of the exterior algebra.
\begin{defi}[tensor algebra, symmetric algebra, exterior algebra]
    If $V$ is a finite dimensional vector space, we define the tensor algebra
    $\cT(V)$ as 
    \begin{equation*}
        \cT(V) = K \oplus V \oplus (V \otimes V) \oplus (V \otimes V \otimes V) + \dots
        = \bigoplus_{n \in \N} V^{\otimes n}.
    \end{equation*}
    The tensor algebra carries the structure of a non-commutative $K$-algebra with
    multiplication given by taking tensors. 

    The \emph{symmetric algebra} $\Sym(V)$ is defined as the universal
    commutative algebra under $\cT(V)$, which is simply the quotient
    \begin{equation*}
        \Sym(V) = \cT(V)/(v \otimes w - w\otimes v \mid v,w \in V).
    \end{equation*}
    Note that $K[V] = \Sym(V^*)$.

    The \emph{exterior algebra} $\bigwedge(V)$ is the universal anti-commutative
    algebra under $\cT(V)$, which is 
    \begin{equation*}
        \bigwedge(V) = \cT(V)/(v \otimes w + w\otimes v \mid v,w \in V).
    \end{equation*}
\end{defi}

Note that both $\Sym(V)$ and $\bigwedge(V)$ have a natural grading inherited from
$\cT(V)$. 
We say that a graded $K$-algebra $R$ is generated in degree $d$ iff it is
generated by $R_{\leq d} = R_1 \oplus \dots \oplus R_d$ as (non-commutative)
$K$-algebra.  That is, the aim of the talk is to show the following 
\begin{thm}[Noether's degree bound for the exterior algebra]\label{thm:mainthm}
    Let $G$ be a finite group of order $d$ acting on a finite vector space $V$.
    This induces an action of $G$ on the exterior algebra $R = \bigwedge V$ of $V$.
    Its subalgebra of $G$-invariant elements $R^G$ is generated in
    degree at most $d$.
\end{thm}

\subsection{A few examples} % (fold)
\label{sub:Some examples}
If $V \cong K^n$, there is a (non-canonical) isomorphism
\begin{equation*}
    \bigwedge(V) \cong \frac{K\langle x_1, \dots, x_n \rangle}{(x_i x_j + x_j
    x_i \mid 1 \leq i,j \leq n)}.
\end{equation*}
Here, $K\langle x_1, \dots, x_n \rangle$ denotes the (non-commutative) algebra generated
by the words in the symbols $x_1, \dots, x_n$.
For an integer $k \geq 0$, we write $\bigwedge{}^k(V)$ for the degree-$k$-piece of 
$\bigwedge(V)$. Note that $\bigwedge{}^k = 0$ if $k > \dim(V)$. 

\textbf{Example 1.} Theorem \ref{thm:mainthm} is trivial if the order of $G$ is greater
or equal to the dimension of $V$. Indeed, in this case the pieces of
$\bigwedge(V)^G \subset \bigwedge(V)$ of degree $>\# G$ vanish, hence
everything is generated in degree $\leq \# G$.

\textbf{Example 2.} Consider the group $G = \Z/4\Z$ and let $G$ act on 
$V = K^4$, with basis $(e_1, \dots, e_4)$ via shifting these basis vectors.
Let $H = \langle 2 \rangle \subset G$. We want to compute the invariants of
$\bigwedge(V)$ under $H$. One quickly finds:
\begin{itemize}
    \item The degree $1$ pieces are ($K$-linearly) generated by $\langle x_1 +
        x_3, x_2 + x_4 \rangle$.
    \item The degree $2$ pieces are ($K$-linearly) generated by $\langle x_1
        x_2 + x_3x_4, x_1 x_3 + x_2 x_4, x_1x_3+x_3x_1,  x_2x_4 + x_4x_2
        \rangle.$
\end{itemize}
Now Noether's degree bound steps in and states that these elements generate
$\bigwedge(V)$! \red{Maybe it would be cool to verify this without Noether's degree
bound in this situation.}

\textbf{Example 3.} 
As it turns out, Noether's bound is false for general non-commutative algebras. 
Now take the algebra $R = K\langle x,y \rangle/(xy+yx)$.
Let $G = \Z/2\Z$ and consider the action of $G$ on $R$ that swaps $x$ and $y$. Now let's
take a look at the invariants. 
\begin{itemize}
    \item The degree $1$ piece of $R^G$ is given by $\{ax + ay | a \in K\}$.
    \item The degree $2$ piece of $R^G$ is given by $\{ax^2 + ay^2 | a \in K\}$.
\end{itemize}
Note that $ax^2 + ay^2 = (ax+ay)(x+y)$, so that the degree-$1$ and degree-$2$ pieces
are generated by $(x+y)$. If Noether's bound were to hold in this setting, we'd
be done at this point. We'd have $R^G = K\langle x+y \rangle$. But there is the
degree-$3$ element $x^3+y^3 \in R^G$, and this is not divisible by $x+y$!
Indeed, if $(x+y)f = x^3 +y^3$, we'd find that $f \in R^G_2$, and necessarily
$f = x^2 + y^2$. But $(x+y)(x^2 + y^2) = x^3 + xy^2 + yx^2 + y^3 \neq x^3 + y^3.$
% subsection A counterexampleme (end)

\subsection{Outline of the talk} % (fold)
\label{sub:Outline of the talk}
As explained above, we want to explain the proof of theorem \ref{thm:mainthm},
which has recently been given in a paper by Francesca Gandini 
\cite{gandini2021degree}. It makes use of an approach to the classical 
problem by Derksen and levereges structural similarities between the algebras
$\bigwedge(V^*)$ and $K[V] = \Sym(V^*)$ to transfer the proof to the 
exterior algebra. In the first half of the talk we will discuss 
the approach of Derksen, in the second we will investigate how Gandini manages
to transfer this approach to the new setting.
% subsection Outline of the talkd)


\end{document}

\documentclass[../main.tex]{subfiles}

\begin{document}
Let $K$ be a field of characteristic $0$, let $G$ be a finite group of order $d$
and let $V$ be a $K$-vector space on which $G$ acts. In previous talks, we have 
seen a proof of the following theorem.

\begin{thm}[Noether's degree bound] \label{thm:classical}
    In the situation above, the action of $G$ of $K[V] = \Sym(V^*)$ has $K[V]^G$ generated
    in degree $\leq d$, where $d$ is the order of $G$.
\end{thm}

In Daniel's talk, we have even seen that this statement can be generalized a bit,
it holds under the milder assumption that $d < \chr K$. 

Today, we want to investigate further generalizations of this result. 
For one, we want to explain an approach by Derksen \cite{DERKSENProofIdea} that
yields Noether's degree bound under the even milder assumption that $d \in
K^\times$. Afterwards, we explain how this approach makes it possible to
proof an analogue of Noether's degree bound for the exterior algebra.

We say that a graded
$K$-algebra $R$ is generated in degree $d$ iff it is generated by $R_{\leq d} =
R_1 \oplus \dots \oplus R_d$ as $K$-algebra.  
That is, the aim of the talk is to show the following 

\begin{thm}[Noether's degree bound for the exterior algebra]\label{thm:mainthm}
    Let $G$ be a finite group of order $d$ acting on a finite vector space $V$.
    This induces an action of $G$ on the exterior algebra $R = \bigwedge V$ of $V$.
    Its subalgebra of $G$-invariant elements $R^G$ is generated in
    degree at most $d$.
\end{thm}

%\subsection{A counterexample} % (fold)
%\label{sub:A counterexample}
%As it turns out, Noether's bound is false for general non-commutative algebras. 
%Now take the algebra $R = K\langle x,y \rangle/(xy+yx)$.
%Let $G = \Z/2\Z$ and consider the action of $G$ on $R$ that swaps $x$ and $y$. Now let's
%take a look at the invariants. 
%\begin{itemize}
%    \item The degree $1$ piece of $R^G$ is given by $\{ax + ay | a \in K\}$.
%    \item The degree $2$ piece of $R^G$ is given by $\{ax^2 + ay^2 | a \in K\}$.
%\end{itemize}
%Note that $ax^2 + ay^2 = (ax+ay)(x+y)$, so that the degree-$1$ and degree-$2$ pieces
%are generated by $(x+y)$. If Noether's bound were to hold in this setting, we'd
%be done at this point. We'd have $R^G = K\langle x+y \rangle$. But there is the
%degree-$3$ element $x^3+y^3 \in R^G$, and this is not divisible by $x+y$!
%Indeed, if $(x+y)f = x^3 +y^3$, we'd find that $f \in R^G_2$, and necessarily
%$f = x^2 + y^2$. But $(x+y)(x^2 + y^2) = x^3 + xy^2 + yx^2 + y^3 \neq x^3 + y^3.$
%This is part of a family of counterexamples. 
%% subsection A counterexampleme (end)

\subsection{Outline of the talk} % (fold)
\label{sub:Outline of the talk}
As explained above, we want to explain the proof of theorem \ref{thm:mainthm},
which has recently been given in a paper by Francesca Gandini 
\cite{gandini2021degree}. It makes use of an approach to the classical 
problem by Derksen and levereges structural similarities between the algebras
$\bigwedge V$ and $K[V] = \Sym(V^*)$ to transfer the proof to the 
exterior algebra. In the first half of the talk we will discuss 
the approach of Derksen, in the second we will investigate how Gandini manages
to transfer this approach to the new setting.
% subsection Outline of the talkd)


\end{document}

\documentclass[../main.tex]{subfiles}

\begin{document}

\section{Derksen's proof for Noether's degree bound}
The Idea behind Gandini's proof of theorem \ref{thm:mainthm} is to
imitate a proof of Noether's degree bound for algebras of the form 
$K[V]$. Again, $V$ is a representation over $K$ of some finite group $G$
whose order is invertible in $K$. 


The approach described here was discovered by Harm Derksen in
\cite{DERKSENProofIdea}, where he proofs propositions \ref{prop:obs1} and 
\ref{prop:obs2}, and states \ref{thm:crux} as a conjecture. The main realization is
that certain subspace arrangements carry all the information about the Hilbert ideal
associated to a group action, and that understanding the Hilbert ideal suffices to
proof Noether's bound. We will review these notions and go on to explain the 
approach.

\begin{defi}[Subspace arrangement]
    Let $W$ be a finite dimensional $K$-vector space. A \emph{subspace
    arrangement} is a finite set of linear subspaces of $W$, denoted by $\cA =
    \{W_1, \dots, W_t\}$. 
\end{defi}

\begin{defi}[Vanishing ideal]
    If $S \subset V$ is any subset of a vector space, we define the associated
    \emph{vanishing ideal}
    $I(S) = \{f \in K[V] \mid \forall s \in S: f(s) = 0\}$.
    Let $W$ and $\cA$ be as above. In this case, the vanishing ideal $I(\cA)$
    is defined as $I(W_1 \cup \dots \cup W_t)$, i.e.,
    \begin{equation*}
        I(\cA) = \{f \in K[V] \mid \forall x \in \cA: f(x) = 0\} =
        \bigcap_{i=1}^t I(W_i).
    \end{equation*}
\end{defi}

\begin{defi}[Subspace Arrangement associated to a group action]
    Let $G$ be a finite group acting on a finite dimensional $K$ vector space $K$, 
    and denote this action by $\pi$.
    We define the subspace arrangement $\cA_G$ associated to this action via
    \begin{equation*}
        \cA_G \coloneqq \bigcup_{g \in G}\{(v,\pi(g)v) \mid v \in V\}
        \subseteq V \oplus V.
    \end{equation*}
\end{defi}
We denote the vanishing Ideal associated to $\cA_G$ by $I(\cA_G) \subset K[V \oplus V]$. 
The key observation behind Derksen's proof is that $I(\cA_G)$ is related to the Hilbert ideal
$J_G \subset K[V]$, which is essentially the ideal in $K[V]$ "generated by $K[V]^G$".

\begin{defi}[Hilbert Ideal]
    The Hilbert ideal is defined as the ideal of $K[V]$ generated by the 
    $G$-invariants of positive degree, i.e., the ideal $J = (K[V]^G_+) K[V] 
    \subset K[V]$. 
\end{defi}
Let $\cR: K[V] \to K[V]^G$ denote the \emph{Reynolds operator}, which is the $K$-vector
space homomorphism given by $f \mapsto \allowbreak \frac{1}{\# G} \allowbreak
\sum_{g \in G} g.f$. 

Derksen's approach consists of three key observations, which we'll state now.
The first observation explains why the Hilbert ideal is interesting for us.
\begin{prop}\label{prop:obs1}
    Suppose the Hilbert ideal is generated by elements $h_1, \dots, h_r \in J$ 
    (note that these are not necessarily $G$-invariant).
    We can assume that all those functions are homogenous. Now the subring of invariants
    is generated by $\cR(h_1), \dots, \cR(h_r)$ over $K$. In formulas:
    \begin{equation*}
        K[V]^G = K[\cR(h_1), \dots, \cR(h_r)].
    \end{equation*}
\end{prop}
In particular, if we can show that the Hilbert ideal is generated in degree $\leq \#G$,
we are done.

The second observation explains why the vanishing ideal of the subspace arrangement
$\cA_G$ is interesting for us. 
\begin{prop}\label{prop:obs2}
    As $K[V\oplus V]$ is graded and noetherian, we can assume that $I(\cA_G) \subset
    K[V \oplus V]$ is generated by homogenous elements
    $$f_1(\xx, \yy), \dots, f_r(\xx,\yy) \in K[V \oplus V].$$ Given such a tuple of 
    elements, the Hilbert ideal is generated by the elements
    \begin{equation*}
        f_1(\xx, 0), \dots, f_r(\xx,0) \in K[V].\footnote{The notation $f(\xx,\yy)$ 
        makes it seem like $\xx$ and $\yy$ are coordinates. This really only 
        makes sense once we chose a basis for $V$, which
        is something one could wish to avoid. To remedy this, we could take
        "abstract" elements $f_i \in K[V \oplus V]$, and write $\pi(f)$ 
        instead of $f(\xx,0)$, where $\pi: K[V \oplus V] \to K[V]$ is the
        $K$-algebra obtained from the first factor projection $V \oplus V \to
        V$ by functoriality.} 
    \end{equation*}
    This statement can equivalently be stated as
    \begin{equation*}
        (I(\cA_G) + (\yy)) \cap K[\xx] = J.
    \end{equation*}
\end{prop}
Hence, in order to finish the proof of theorem \ref{thm:classical} it suffices
to show that $I(\cA_G)$ is generated in degree $\leq d$. Note that $I(\cA_G)$
is the intersection of $d$ linear ideals in a ring isomorphic to $K[x_1, \dots,
x_n]$, so this at least seems plausible: In the "extreme" case where
$\bigcap_{g \in G} I(V_g) = \prod_{g \in G} I(V_g)$, which for example is the
case if the have pairwise trivial intersection,\footnote{This applies the neat
criterion described in \cite{productofidealsintersection}} this statement is
clear: The product is generated by $d$-fold products of linear polynomials. In
general, questions like this are hard to answer. 
But in this situation, Derksen and Sidman were able to provide an
answer in \cite{DERKSENRegularity}.
\begin{thm}\label{thm:crux}
    Let $\cA = \{W_1, \dots, W_t\}$ be a subspace arrangement. Then 
    $I(\cA)$ is generated in degree $\leq t$.
\end{thm}
Technically, they show the stronger statement that $I(\cA)$ is $t$-regular, but 
we will not explain this notion here. See chapter 20.5 of
\cite{eisenbud2013commutative} for an introduction.

\subsection{Proof of proposition 8} % (fold)
\label{sub:Proofs of proposition 8}
We will prove this is using a few lemmas.
Remember that the Hilbert ideal was denoted by $J$.
\begin{lem}\label{lem:8.1}
    Suppose that $J$ is generated by elements $f_1, \dots, f_r \in K[V]$. Then
    the elements
    $\cR(f_1), \dots, \cR(f_r)$ generate $J$.
\begin{proof}
    Remember that $J$ was the ideal in $K[V]$ generated by $K[V]^G_+$. 
    Denote $$J' \coloneqq (\cR(f_1), \dots, \cR(f_r)).$$ We have $\cR(f_i) \in
    K[V]^G\subset J$ and in particular $J' \subset J$. We want to show the
    reverse inclusion. As a first step, we show that for any 
    $a \in K[V]$ and any $g \in G$, it holds that $g(a) - a \in K[V]^+$. 
    Indeed, suppose $g(a) - a = x \in K\setminus\{0\}$. We find $g(a) = a + x$, hence
    $a = g^d(a) = a + dx \neq a$. Here we used that $d \in K^\times$! This small
    result implies that $G$ acts trivially on the quotient $J/K[V]_+ J$, and in particular,
    the residue class of $f_i$ is the same as that of $\cR(f_i)$. Hence, 
    $J' + K[V]_+ J = J$. But now we compute
    \begin{equation*}
        K[V]_+(J/J') = (K[V]_+ J + J')/J' = J/J',
    \end{equation*}
    readily implying $J = J'$.
\end{proof}

\begin{lem}
    Let $R$ be a graded $K$-algebra and suppose that $R_{+}$ is generated by
    as ideal by homogenous elements $f_1, \dots, f_r \in R_+$. Then $R = K[f_1,
    \dots, f_r]$.
\begin{proof}
    It is clear that $K[f_1, \dots, f_r] \subset R$, we show the reverse inclusion.
    Let $g \in R_+$ be an arbitrary homogenous element. 
    We want to show $g \in K[f_1, \dots, f_r]$, via induction on the degree of $g$.
    The base case $\deg g = 0$ is clear. Let's assume $\deg g > 0$. By assumption,
    we can write $g = \sum_{i=1}^r a_i f_i$. Now (after reduction) all
    non-trivial summands satisfy $\deg(a_i) + \deg(f_i) = \deg g$. As
    $\deg(f_i) > 0$, we find $\deg(a_i) < \deg (g)$, so $a_i \in K[f_1, \dots, f_r]$ 
    for all $i$. This implies $g \in K[f_1, \dots, f_r]$, as required.
\end{proof}
\end{lem}

\end{lem}
As a corollary, we obtain
\begin{lem}\label{lem:8.3}
    Assume that the Hilbert ideal $J$ is generated by homogenous and
    $G$-invariant elements $f_1, \dots, \allowbreak f_r \in K[V]^G$. Then
    $K[V]^G = K[f_1, \dots, f_r]$. 
\begin{proof}
    One inclusion is clear. Let $g \in K[V]^G$, we show that $g \in K[f_1, \dots, f_r]$.
    It suffices to show that $g = \sum f_i a_i$ with $a_i \in K[V]^G$
    by the lemma above (applied with $R = K[V]^G$). As $K[V]^G \subset J$, we
    have a representation $g = \sum f_i a_i$ with $a_i \in K[V]$. Now apply
    $\cR$, obtaining $g = \cR(g) = \sum f_i \cR(a_i)$. This is the expression
    we need.
\end{proof}
\end{lem}

Now proposition \ref{prop:obs1} follows quickly. 
Applying Lemma \ref{lem:8.1}, we replace the generators $f_1, \dots, f_r$ of
$J$ with their respective images under the Reynolds operator $\cR(f_1), \dots,
\cR(f_r)$. Subsequently, employing Lemma \ref{lem:8.3} completes the argument.

% subsection subsection name (end)
\subsection{Proof of proposition 9} % (fold)
\label{sub:Proof of proposition 9}
To ease notation, we'll write $K[V \oplus V]$ as $K[\xx,\yy]$. Here $G$ acts trivially
on the $\xx$-part and as usual on the $\yy$-part.
Let $J' = (f_1(\xx, 0), \dots, f_r(\xx,0))$. We have to show that $J' = J$, and
we show both inclusions separately. The easier one is $J \subset J'$. 
As $J$ is generated by homogeneous $G$-invariant objects, it suffices to show that any
such $g \in K[V]^G$ lies in $J'$. As $g$ is $G$-invariant, we verify $g(\xx)-g(\yy) \in
I(\cA_G)$ (indeed, $g(\xx)-\gamma(\sigma(\xx)) = 0$ for all $\sigma \in G$). In particular, $g(\xx) - g(\yy) = \sum_{i=1}^r a_i(\xx,\yy) f_i(\xx,\yy)$,
and the claim follows, as
\begin{equation*}
    g(\xx) = g(\xx) - g(0) = \sum_{i=1}^r a_i(\xx,0) f_i(\xx,0) \in J'.
\end{equation*}

Now for the reverse inclusion. The Reynolds operator for the action of $G$ on 
$K[\xx ,\yy]$ is a morphism of $K[\xx]$-modules
\begin{equation*}
    K[\xx,\yy] \to K[\xx, \yy]^G \cong K[\xx] \otimes_K K[\yy]^G.
\end{equation*}
We also define morphism of algebras
\begin{equation*}
    \delta: K[\xx,\yy] \to K[\xx], \yy \mapsto \xx.
\end{equation*}
Let $f(\xx,\yy) \in I(\cA_G)$ be an arbitrary element. We want to show that 
$f(\xx,0)$ lies in the Hilbert ideal, i.e., that there is a linear combination
$f(\xx, 0) = \sum c_i(\xx) h_i(\xx)$ with $G$-invariant elements $h_i \in K[V]^G$. 
Note that $f(\xx,0) = f(\xx,\yy) - r(\xx,\yy)$, where $r(\xx,\yy)$ is an element
in the ideal $(\yy) \subset K[\xx,\yy]$ and can thereby be written as $\sum
c_i(\xx)p_i(\yy)$ with $p_i(0) = 0$. The next step is tricky: Applying the
Reynolds operator, we find 
\begin{equation*}
    f(\xx, 0) = \cR(f(\xx,0)) = \cR(f(x,y)) - \sum_i c_i(\xx) \cR(p_i(\yy)).
\end{equation*}
Now applying $\delta$ gives
\begin{equation*}
    f(\xx,0) = \delta \cR(f(\xx,0)) = \sum_i c_i(\xx) \delta \cR(p_i(\yy)).
\end{equation*}
Here we are done, as $\delta \cR(p_i(\yy))$ is simply $\cR(p_i(\xx))$ with the 
usual Reynolds operator on $K[V]$, and in particular $G$-invariant. 

This finishes the proof of proposition \ref{prop:obs2}, and with that of the
classical degree bound, theorem \ref{thm:classical}.
% subsection subsection name (end)
\end{document}
